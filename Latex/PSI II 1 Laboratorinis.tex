\documentclass[oneside]{VUMIFPSkursinis}
\usepackage{algorithmicx}
\usepackage{algorithm}
\usepackage{algpseudocode}
\usepackage{amsfonts}
\usepackage{float}
\usepackage{amsmath}
\usepackage{bm}
\usepackage{caption}
\usepackage{color}
\usepackage{float}
\usepackage{graphicx}
\usepackage{listings}
\usepackage{subfig}
\usepackage{ltablex}
\usepackage{longtable}
\usepackage{wrapfig}
\newcolumntype{P}[1]{>{\centering\arraybackslash}p{#1}}
\usepackage[%  
    colorlinks=true,
    linkcolor=black
]{hyperref}
\university{Vilniaus universitetas}
\faculty{Matematikos ir informatikos fakultetas}
\department{Programų sistemų katedra}
\papertype{Programų sistemų inžinerija II laboratorinis darbas I}
\title{Reikalavimų apibrėžimas}
\titleineng{Requirements specification}
\status{2 kurso 3 grupės studentai}
\secondauthor{Justas Tvarijonas}  
\thirdauthor{Greta Pyrantaitė}
\fourthauthor{Rytautas Kvašinskas}
\author{Tomas Kiziela}

\supervisor{Audronė Lupeikienė, M. Darbuot., Dr.}
\date{Vilnius – \the\year}

\bibliography{bibliografija}

\begin{document}
\maketitle
\tableofcontents

\section{Įžanga}
Mūsų komanda gavo kitos komandos pirmame semestre(PSI I) paruoštą slidinėjimo kurorto projektą. Šiame darbe sieksime toliau plėtoti ir keisti šį projektą. Toliau vystatnt projektą keisis daugumas dalių. Siekiant padaryti gerą produktą kai kurios  dalys bus pašalindos ir pridėtos naujos. Pirmame semestre projektuojant dėmesys buvo skiriamas klasikinei projektavimo paradigmai. Šiame darbe projektą rašysime pasinaudodami ICONIX principu, projektuojant dėmesys bus skiriamas GUI sudarymo ir iš to išplauks sistemos projektavimas ir sandara. 

\section{Reikalavimai}
Šioje dalyje bus pateikti funkciniai bei nefunkciniai reikalavimai sistemai. Stengsimės prisilaikyti ,,užsakovų"(grupės iš kurios gavome jų darbą) reikalavimus tačiau siekiant sukurti geresnę sistemą pridėsime kaikuriuos savo sugalvotus reikalavimus arba ignoruosime mums pateiktus reikalavimus. 

\subsection{Reikalavimų pataisymai}
	\begin{itemize}
		\item{X - ištrintas}
		\item{M - pakeistas}
	\end{itemize}

%\begin{table}[htbp]

\begin{longtable}{ | p{0.04\textwidth}|p{0.43\textwidth}|p{0.09\textwidth}|p{0.15\textwidth}|p{0.21\textwidth}| }  \hline
	Nr. & Pradiniai reikalavimas&  Pakeitimo tipas & Klaidos aprašas  & Naujas reikalavimas \\ \hline
	FR & ,,Žetonas" seka vartotojo laiką praleistą trasoje & M & Pakeistas neaiškus daiktavardis & Sekimo prietaisas seka vartotojo laiką praleista trasoje \\ \hline	
	FR & Vartotojo prieeigos prie pramogų prieinamumas nustatomas naudojant pirštų antspaudą & M & Sukonkretintas abstraktus reikalavimas & Vartotojui norint pasinaudoti kavinės paslaugomis naudojamas piršto antspaudas \\ \hline
	FR &  Sistema leidžia vartotojui už paslaugas atsiskaityti e-bankininkyste & N & - & - \\ \hline
	FR & ,,Žetonas" skaičiuoja greičiausią laiką, per kurį vartotojas įveikia trasą & M & Pakeistas neaiškus daiktavardis & Sekimo prietaisas fiksuoją greičiausią laiką, per kurį vartotojas įveikė trasą \\ \hline
	FR & Vartotojo trasų laikai rodomi internetinėje aplikacijoje & N & NaN & NaN \\ \hline
	FR & Sistema seka vartotojų poziciją specialaus žetono pagalba, kurį gauna kiekvienas vartotojas apsilankęs kurorte(Vieta sekama tik gavos vartotojo sutikimą) & M & Reikalavimas skliausteliuose turėtų būti perkeltas į nefunkcinius (pvz. saugumo) raiklavimus & Sistema seka vartotojo poziciją specialaus sekimo prietaiso pagalba, kurį gauna kiekevienas apsilankęs kurorte \\ \hline
	FR & 3.1.6. Atsijungimas
FR 26. Pradiniai duomenys
FR 26.1. Nėra
FR 27. Veiksmai
FR 27.1. Vartotojas meniu juostoje paspaudžia ant mygtuko “Log out”;
FR 28. Alternatyvūs scenarijai
FR 28.1. Jei vartotojas yra rezervacijos lange, jo paklausiama ar tikrai nori atsijungti
FR 29. Reikalavimai
FR 29.1. Vartotojas turi būti prisijungęs;
FR 30. Rezultatai
FR 30.1. Vartotojas atsijungia ir jam parodomas pirminis langas & M & Apibendrinamas reikalavimas & Vartotojas turi galėti atsijungti nuo sistemos\\ \hline
FR & 3.1.7. Atnaujinti slaptažodį
FR 31. Pradiniai duomenys
FR 31.1. E-paštas
FR 32. Veiksmai
FR 32.1. Vartotojas meniu juostoje paspaudžia ant mygtuko “Forgot Password”;
FR 32.2. Vartotojas suveda savo E-paštą ir paspaudžia ant mygtuko “New Password”;
FR 32.3. Vartotojui išsiunčiamas naujas sugeneruotas slaptažodis
FR 33. Reikalavimai
FR 33.1. Vartotojas turi būti atsijungęs;
FR 34. Rezultatai
FR 34.1. Vartotojas parodomas prisijungimo puslapis & M & Apibendrinamas reikalavimas & Vartotojas turi galėti atnaujinti slaptažodį\\ \hline
	FR & 3.2.1. Pasižiūrėti orus
FR 35. Pradiniai duomenys
FR 35.1. Data;
FR 36. Veiksmai
FR 36.1. Vartotojas meniu juostoje paspaudžia ant mygtuko „Weather“;
FR 36.2. Vartotojas pasirenka dieną, kurios orų prognozę nori sužinoti;
FR 37. Alternatyvūs scenarijai
FR 37.1. Jei data mažesnė, nei dabartinė diena, parodoma klaida;
FR 37.2. Jei orų prognozės nėra duomenų bazėje, parodomas pranešimas;
FR 38. Reikalavimai
FR 38.1. Data turi būti ne mažesnė nei dabartinė diena;
FR 38.2. Orų prognozė pasirinktai dienai turi būti duomenų bazėje;
FR 39. Rezultatai
FR 39.1. Vartotojui pateikiama orų prognozė; & M & Apibendrinamas reikalavimas & Vartotojas turi galėti pasižiūrėti orus\\ \hline
	FR & 3.2.2. Parašyti laišką
FR 40. Pradiniai duomenys
FR 40.1. Laiško tema;
FR 40.2. Laiško turinys;
FR 41. Veiksmai
FR 41.1. Vartotojas meniu juostoje paspaudžia ant mygtuko “Mail”;
FR 41.2. Vartotojas parašo laiško temą bei turinį;
FR 41.3. Vartotojas paspaudžia mygtuką “Send”
FR 42. Alternatyvūs scenarijai
FR 42.1. Jei vartotojas neprisijungęs, išmetamas prisijungimo puslapis;
FR 42.2. Jei tema ar turinys tuščias, laukelis paraudonuoja ir laiškas nėra išsiunčiamas;
FR 43. Reikalavimai
FR 43.1. Vartotojas turi būti prisijungęs;
FR 43.2. Turinys bei tema neturi būti tušti;
FR 43.3. Temos ilgis neturi viršyti 50 simbolių;
FR 43.4. Turinio ilgis neturi viršyti 500 simbolių;
FR 44. Rezultatai
FR 44.1. Vartotojui pranešama, jog laiškas sėkmingai išsiųstas; & M & Apibendrinamas reikalavimas & Vartotojas turi galėti parašyti laišką \\ \hline

	FR & 3.3.1. Parašyti ataskaitą
FR 45. Pradiniai duomenyswhy 
FR 45.1. Ataskaitos pavadinimas;
FR 45.2. Ataskaitos turinys;
FR 46. Veiksmai
FR 46.1. Darbuotojas meniu juostoje paspaudžia ant mygtuko “Reports”;
FR 46.2. Darbuotojas parašo ataskaitos pavadinimą bei turinį;
FR 46.3. Darbuotojas paspaudžia mygtuką “Send”
FR 48. Reikalavimai
FR 48.1. Darbuotojas turi būti prisijungęs;
FR 48.2. Pavadinimas bei tema neturi būti tušti;
FR 48.3. Pavadinimo ilgis neturi viršyti 50 simbolių;
FR 48.4. Turinio ilgis neturi viršyti 500 simbolių;
FR 49. Rezultatai
FR 49.1. Darbuotojui pranešama, jog ataskaita sėkmingai išsiųstas. & M & trying to get gold from shit & Vartotojas turi galėti parašyti ataskaitą\\ \hline

Nr. & Pradiniai reikalavimas&  Pakeitimo tipas & Klaidos aprašas  & Naujas reikalavimas \\ \hline
FR & 
FR 93. Pradiniai duomenys: 
FR 93.1. Objekto kategorija(-os). 
FR 94. Veiksmai: 
FR 94.1. Meniu juostoje paspaudžiama ant mygtuko „Set the object state”;
 FR 94.2. Iš objektų kategorijų sąrašo (kambarys, slidinėjimo trasa, slidinėjimo įranga) pasirenkama viena kategorija;
 FR 94.3. Parodomas sąrašas objektų, priklausančių tai kategorijai;
 FR 94.4. Pasirenkamas vienas objektas; FR 94.5. Pasirenkama data;
 FR 94.6. Pažymima būsena, paspaudžiant ant mygtuko „Set the state”.
 FR 95. Alternatyvūs scenarijai:
 FR 95.1. Nutrūko ryšys su duomenų baze.
 FR 96. Reikalavimai: 
FR 96.1. Administratorius turi būti prisijungęs.
 FR 97. Rezultatai:
 FR 97.1. Duomenų bazėje pažymima objekto būseną. 
& M & Reikalavimai parašyti nekorektiškai, reikalinga sukonkretinti ir suabstraktinti reikalavimus į vieną reikalavimą &
Sistema administratoriui suteikia galimybę stebėti duombazės objektų būseną. Sistemoje administratorius gali stebėti arba keisti objektų būsenas. Pasirinkus veikmsą parodomas objektų kategorijų sąrašas. Nutrūkus ryšiui su duombaze yra išsaugoma paskutinė darbo būsena.\\ \hline
FR & 3.4.4. Peržiūrėti statistiką
FR 58. Pradiniai duomenys
FR 58.1. Laikotarpis, kurio duomenis savininkas nori gauti;
FR 58.2. Duomenų tipas (klientų srautas ir/arba pajamos)
FR 59. Veiksmai
FR 59.1. Pasirinkti duomenų tipą;
FR 59.2. Pasirinkti laikotarpį;
25
FR 59.3. Spausti mygtuką „Get data”
FR 60. Alternatyvūs scenarijai:
FR 60.1. Pasirinktam laikotarpiui nėra įrašų duomenų bazėje, tada išvedamas pranešimas apie tai.
FR 61. Reikalavimai:
FR 61.1. Savininkas privalo būti prisijungęs prie savo paskyros;
FR 61.2. Būtina pasirinkti egzistuojantį laikotarpį ( tai reiškia negali data “nuo” būti vėliau negu data “iki”);
FR 62. Rezultatas:
FR 62.1. Diagramoje pavaizduojami pasirinkti duomenys. & M &  Reikalavimai parašyti nekorektiškai, reikalinga sukonkretinti ir suabstraktinti reikalavimus į vieną reikalavimą & vartotojas turi galėti peržiūrėti statistiką. \\ \hline
FR & 3.4.5. Peržiūrėti paslaugų ar įrangos tiekėjų sąrašą
FR 63. Pradiniai duomenys
FR 63.1. Paslaugų tipas(-ai) (įranga, elektros energija, šilumos energija, apsaugos tarnyba);
FR 63.2. Laikotarpis, kurio duomenis savininkas nori gauti;
FR 64. Veiksmai:
FR 64.1. Pasirinkti paslaugų tipą(-us);
FR 64.2. Pasirinkti laikotarpį (gali būti tuščias);
FR 64.3. Spausti mygtuką „See suppliers”;
FR 65. Alternatyvūs scenarijai:
FR 65.1. Pasirinktai paslaugai nėra apibrėžtas tiekėjas, tada išvedamas pranešimas apie tai;
FR 66. Reikalavimai:
FR 66.1. Savininkas privalo būti prisijungęs prie savo paskyros;
FR 66.2. Būtina pasirinkti bent vieną paslaugą;
FR 66.3. Paslaugos privalo būti pavaizduotos drop-down sąraše;
FR 66.4. Kol nepasirinkta nė viena paslauga, mygtukas yra išjungtas;
FR 67. Rezultatas:
FR 67.1. Pavaizduojamas sąrašas pasirinktų paslaugų tiekėjų. & M &  Reikalavimai parašyti nekorektiškai, reikalinga sukonkretinti ir suabstraktinti reikalavimus į vieną reikalavimą & vartotojas turi galėti peržiūrėti paslaugų ir įrangos tiekėjų sąrašus \\ \hline


FR &  FR 68. Pradiniai duomenys:
FR 68.1. Sutarties pavadinimas;
FR 69. Veiksmai:
FR 69.1. Pasirinkti norimą sutartį;
FR 69.2. Spausti mygtuką „Read the agreement”;
FR 69.3. Alternatyvūs scenarijai:
FR 69.4. Pasirinkta sutartis prieš akimirką buvo nutraukta, ir kitas savininkas pažymi apie tai, tada išvedamas pranešimas apie tai;
FR 70. Reikalavimai:
FR 70.1. Savininkas privalo būti prisijungęs prie savo paskyros;
FR 70.2. Sutartys privalo būti pavaizduotos drop-down sąraše;
FR 70.3. Kol nepasirinkta sutartis, mygtukas yra išjungtas;
FR 70.4. Galima pasirinkti tik vieną sutartį. & M & Bandyta dalinti iš nulio su šitais reikalavimai & Vartotojas turi galėti peržiūrėti sutartis. \\ \hline

FR & FR 71. Pradiniai duomenys:
FR 71.1. Vartotojo tipas (klientas, darbuotojas, administratorius, savininkas);
FR 71.2. Administratoriaus el. paštas;
FR 71.3. Slaptažodis.
FR 72. Veiksmai:
FR 72.1. Pasirenkamas prisijungimo tipas;
FR 72.2. Įvedami prisijungimo duomenys;
FR 72.3. Spaudžiamas mygtukas „Log in”.
FR 73. Alternatyvūs scenarijai:
FR 73.1. Įvedus neteisingus duomenis, ar palikus bent vieną tuščią laukelį, langeliai pažymimi raudonai ir išvedamas pranešimas dėl neteisingo duomenų įvedimo.
FR 74. Reikalavimai:
FR 74.1. Administratoriaus el. paštas ir slaptažodis turi sutapti su duomenimis esančiais duomenų bazėje;
FR 74.2. Įvedamas slaptažodis negali būti matomas.
FR 75. Rezultatas:
FR 75.1. Administratorius prisijungia prie savo paskyros;
FR 75.2. Atidaromas pagrindinis puslapis. & M & Kam kilo idėja taip padaryt tokį reikalavimą? & Administratorius turi galėti prisijungti prie savo paskyros. \\ \hline

FR & FR 76. Veiksmai:
FR 76.1. Meniu juostoje paspaudžiama ant mygtuko „Employees”;
FR 77. Alternatyvūs scenarijai:
FR 77.1. Įmonėje niekas nedirba, tada išvedamas pranešimas apie tai.
FR 78. Reikalavimai:
FR 78.1. Darbuotojų sąrašas surikiuotas pareigų svarbos mažėjimo tvarka.
FR 79. Rezultatas:
FR 79.1. Išvedamas darbuotojų sąrašas. & M & Performuluoti ir sutraukti funkciniai reikalavimai & Vartotojas gali peržiūrėti darbuotojų sąrašą. \\ \hline

FR & FR 80. Pradiniai duomenys:
FR 80.1. Darbuotojo ID.
FR 81. Veiksmai:
FR 81.1. Meniu juostoje paspaudžiama ant mygtuko „Modify the list of employees”;
FR 81.2. Paspaudžiamas mygtukas „Add new” ;
FR 81.3. Tekstiniame lauke įrašomas darbuotojo ID;
FR 81.4. Paspaudžiamas mygtukas pridėti.
FR 82. Alternatyvūs scenarijai:
FR 82.1. Darbuotojo ID nėra duomenų bazėje, tada išvedamas pranešimas apie tai ir tekstinis laukas pažymimas raudonai.
FR 83. Reikalavimai:
FR 83.1. Darbuotojo ID turi būti duomenų bazėje;
FR 83.2. Darbuotojų sąrašas turi būti surikiuotas pareigų svarbos mažėjimo tvarka;
FR 83.3. Pridėjus darbuotoją sąrašas iš karto atsinaujina. & M & Performuluoti ir sutraukti funkciniai reikalavimai & Vartotojas turi galimybę pridėti darbuotoją į sistemą. \\ \hline

FR & FR 84. Pradiniai duomenys:
FR 84.1. Darbo sutarties nutraukimo sutartis.
FR 85. Veiksmai:
FR 85.1. Meniu juostoje paspaudžiama ant mygtuko „Modify the list of employees”;
FR 85.2. Paspaudžiamas mygtukas „-”;
FR 85.3. Sąraše pažymimas darbuotoją (galima tik vieną);
FR 85.4. Virš sąrašo pasirodo mygtukas „Remove”;
FR 85.5. Paspaudžiamas virš sąrašo esantis mygtukas „Remove”;
FR 85.6. Po mygtuku „Remove” pasirodo dialogo langas su klausimu „Are you sure?”;
FR 85.7. Spaudžiame mygtuką „Yes”.
FR 85.8. Alternatyvūs scenarijai:
FR 85.9. Dialogo lange paspaudus „No” forma atsinaujina, nė vienas darbuotojas nėra pasirinktas pašalinimui.
FR 86. Reikalavimai:
FR 86.1. Darbuotojų sąrašas turi būti surikiuotas pareigų svarbos mažėjimo tvarka;
FR 86.2. Turi būti papildomas tekstinis laukas, kuris palengvina darbuotojo paiešką (t.y. paieškos laukas);
FR 86.3. Pašalinus darbuotoją iš karto sąrašas atsinaujina.
FR 87. Rezultatas:
FR 87.1. Pasirinktas darbuotojas pašalinamas iš darbuotojų sąrašo. & M & Performuluoti ir sutraukti funkciniai reikalavimai & Vartotojas gali atleisti darbuotojus \\ \hline

FR & FR 88. Pradiniai duomenys:
FR 88.1. Objekto kategorija(-os).
FR 89. Veiksmai:
FR 89.1. Meniu juostoje paspaudžiama ant mygtuko „Check the object state”;
FR 89.2. Iš objektų kategorijų sąrašo (kambarys, slidinėjimo trasa, slidinėjimo įranga) pasirinkti ne mažiau nei vieną kategoriją;
FR 89.3. Pasirinkti datą;
FR 89.4. Paspausti mygtuką „Show states”.
FR 90. Alternatyvūs scenarijai:
FR 90.1. Bent vienoje kategorijoje nėra nė vieno objekto, tada išvedamas pranešimas apie trūkstamus objektus.
FR 91. Reikalavimai:
FR 91.1. Administratorius privalo būti prisijungęs prie paskyros.
FR 92. Rezultatas:
FR 92.1. Lentelėje(-ėse) pavaizduojamos pasirinktų objektų būsenos. & M & I swear to God, whoever came up with these..... & Vartotojas gali peržiūrėti nuomojamų objektų būseną. \\ \hline

FR & FR 93. Pradiniai duomenys:
FR 93.1. Objekto kategorija(-os).
FR 94. Veiksmai:
FR 94.1. Meniu juostoje paspaudžiama ant mygtuko „Set the object state”;
FR 94.2. Iš objektų kategorijų sąrašo (kambarys, slidinėjimo trasa, slidinėjimo įranga) pasirenkama viena kategorija;
FR 94.3. Parodomas sąrašas objektų, priklausančių tai kategorijai;
FR 94.4. Pasirenkamas vienas objektas;
FR 94.5. Pasirenkama data;
FR 94.6. Pažymima būsena, paspaudžiant ant mygtuko „Set the state”.
FR 95. Alternatyvūs scenarijai:
FR 95.1. Nutrūko ryšys su duomenų baze.
FR 96. Reikalavimai:
FR 96.1. Administratorius turi būti prisijungęs.
FR 97. Rezultatai:
FR 97.1. Duomenų bazėje pažymima objekto būseną. & M & Honestly, I'm out of ideas what to put here & Vartotojas gali pakeisti objektų būsenas duomenų bazėje. \\ \hline


FR & 
FR 98. Pradiniai duomenys: 
FR 98.1. Ataskaitos data. 
FR 99. Veiksmas:
 FR 99.1. Meniu juostoje paspaudžiama ant mygtuko „Read the report”;
 FR 99.2. Pasirenkama data (neprivaloma);
 FR 99.3. Paspaudžiama ant mygtuko „Read the report”. 
FR 100. Alternatyvūs scenarijai:
FR 100.1. Nė vienos sutarties nėra duomenų bazėje, tada apie tai išvedamas 
pranešimas. 
FR 101. Reikalavimai: 
FR 101.1. Administratorius yra prisijungęs.
 FR 101.2. Ataskaitų sąrašas, jei nepasirinkta data, surikiuotas pagal datą nuo naujausio iki seniausio.
 FR 101.3. Nepasirinkus nė vienos ataskaitos, mygtukas „Set the state” yra išjungtas. 
FR 101.4. Ataskaitos negalima parsisiųsti (apsauga nuo duomenų nutekėjimo). 
FR 102. Rezultatai:
 FR 102.1. Naujame lange parodoma sutartis. 
& M & Netikslūs ir perdaug konkretūs reikalavimai primenantys nefunkcinius reikalavimus. &
Sistema administratoriui suteikia galimybę skaityti sutarčių ataskaitas. Administratorius pasirenka skaityti ataskaitas ir naujame lange parodomos visos ataskaitos esančios duomenų bazėje. Jeigu duomenų bazėje nei vienos sutarties nėra parodomas informacinis pranešimas. \\ \hline


FR & 
FR 103. Pradiniai duomenys: 
FR 103.1. Kliento duomenys;
 FR 103.2. Rezervacijos užklausos duomenys.
 FR 104. Veiksmai: 
FR 104.1. Patikrinami kliento duomenys;
 FR 104.2. Patikrinami užklausos duomenys;
 FR 104.3. Patvirtinama rezervacija; 
FR 104.4. Duomenų bazėje pažymima objekto būsena (užimta). 
FR 105. Alternatyvūs scenarijai:
 FR 105.1. Gautas pranešimas, kad prieš akimirką objektas, kurį klientas bandė išsinuomoti, tapo netinkamas rezervacijai dėl įvairių priežasčių;
 FR 105.2. Klientas pateikė nevalidžius duomenis, tada rezervacija nepatvirtinama. 
FR 106. Reikalavimai: FR 106.1. Kliento informacija validi;
 FR 106.2. Duomenų bazėje objektas yra pažymėtas kaip tinkamas nuomai ir nerezervuotas.
 FR 107. Rezultatai: 
FR 107.1. Rezervacija patvirtinama;
 FR 107.2. Klientas gauna el. pašte laišką apie sėkmingą rezervaciją.
& M &Reikalavimai išmetyti, neaiškiai surašyti ir kaikurie perdaug konkretų.Reikia sukonkretinti ir sutvarkyti & 
Klientui rezervuojant paslaugas sistema turi patvirtinti rezervacijos korektiškumą. Sistema turi patikrinti ar teisingai įvesti rezervacijos duomenys, įrašyti rezervacijols duomenis į duomenų bazę ir patvirtinti rezervaciją. Paslaugai tapus nebepasiekiamai arba vartotojui įvedus neteisingus duomenis parodomas informacinis pranešimas apie nepavykusia rezervaciją ir priežastis dėl ko rezervuoti nepavyko \\ \hline
FR&
FR 68. Pradiniai duomenys:
FR 68.1. Sutarties pavadinimas;
FR 69. Veiksmai:
FR 69.1. Pasirinkti norimą sutartį;
26
FR 69.2. Spausti mygtuką „Read the agreement”;
FR 69.3. Alternatyvūs scenarijai:
FR 69.4. Pasirinkta sutartis prieš akimirką buvo nutraukta, ir kitas savininkas pažymi apie tai, tada išvedamas pranešimas apie tai;
FR 70. Reikalavimai:
FR 70.1. Savininkas privalo būti prisijungęs prie savo paskyros;
FR 70.2. Sutartys privalo būti pavaizduotos drop-down sąraše;
FR 70.3. Kol nepasirinkta sutartis, mygtukas yra išjungtas;
FR 70.4. Galima pasirinkti tik vieną sutartį. & M & Reikalavimų apibendrinimas & vartotojas turi galėti peržiūrėti sutartis \\ \hline





\end{longtable}
\begin{longtable}{ | p{0.04\textwidth}|p{0.43\textwidth}|p{0.09\textwidth}|p{0.15\textwidth}|p{0.21\textwidth}| }  \hline
	Nr. & Reikalavimai & Prioritetas \\ \hline
	FR 1 & Sistema leidžia vartotojui už paslaugas atsiskaityti e-bankininkyste & 10 \\ \hline
	FR 2 & Vartotojo prieeigos prie pramogų prieinamumas nustatomas naudojant pirštų antspaudą & 8 \\ \hline
	FR 3& Sistema seka vartotojų poziciją specialaus žetono pagalba, kurį gauna kiekvienas vartotojas apsilankęs kurorte(Vieta sekama tik gavos vartotojo sutikimą) & 8  \\ \hline
	FR 4 & ,,Žetonas" seką vartotojo laiką praleista trasoje & 8 \\ \hline
	FR 5 & ,,Žetonas" skaičiuoją greičiausią laiką per kurį vartotojas įveikia trasą &  8 \\ \hline
	FR 6 & Vartotojo trasų laikai rodomi internetinėje aplikacijoje & 7\\ \hline 

\end{longtable}
	
%\end{table}

 
\section{Nefunkciniai}


\begin{longtable}{ | p{0.04\textwidth}|p{0.43\textwidth}|p{0.09\textwidth}|p{0.15\textwidth}|p{0.21\textwidth}| }  \hline

Nr. & Pradiniai reikalavimas&  Pakeitimo tipas & Klaidos aprašas  & Naujas reikalavimas \\ \hline
NFR & NFR 14. Slidinėjimo trasų, įrangos bei kambarių pavadinimams maksimaliai skiriama 64 simboliai. & N & NaN & NaN \\ \hline
NFR & NFR 15. Slidinėjimo trasos ilgis vaizduojamas trijų skaičių po kablelio tikslumu. Ilgio matavimo vienetas - kilometrai.  & N & NaN & NaN \\ \hline
NFR & NFR 16. Slidinėjimo trasos statumas vaizduojamas dviejų skaičių po kablelio tikslumu. Statumo matavimo vienetas - procentai. & N & NaN & NaN \\ \hline
NFR & NFR 17. Slidinėjimo trasų, įrangos bei apgyvendinimo įstaigos laisvų vietų kiekis rodomas vienetų tikslumų. & N & NaN & NaN \\ \hline
NFR & NFR 18. Slidinėjimo trasų, įrangos bei apgyvendinimo įstaigos nuomos kainos pateikiamos euro centų tikslumu. & N & NaN & NaN \\ \hline
NFR & NFR 19. Data turi būti vaizduojama formatu YYYY-MM-DD, kur YYYY - metai, MM - mėnuo, DD - diena. & N & NaN & NaN \\ \hline
NFR & NFR 20. Laikas turi būti vaizduojamas formatu: hh:mm, kur hh - valandos, mm - minutės. & N & NaN & NaN \\ \hline
NFR & NFR 21. Esybės vardui, pavardei, elektroniniam paštui, slaptažodžiui registracijos formoje maksimaliai skiriama 64 simboliai. Taip pat registracijos formoje 
esybėms (išskyrus klientus) reikia įvesti raktą, kuriam maksimaliai skiriama 32 simbolių. & N & NaN & NaN \\ \hline
NFR & NFR 22. Esybės raktas sugeneruojas pagal GUID. & N & NaN & NaN \\ \hline
NFR & NFR 23. Esybės elektroniniam paštui ir slaptažodžiui prisijungimo formoje įvesti maksimaliai skiriama 64 simboliai. & N & NaN & NaN \\ \hline
NFR & NFR 24. Vartotojo slaptažodis turi būti nuo 10 iki 64 simbolių. Jame turi būti panaudota bent po vieną didžiąją, mažąją raidę, skaitmenį ir/ar kitą simbolį. & N & NaN & NaN \\ \hline
NFR & NFR 25. Vardui, pavardei, elektroniniam paštui rezervacijos formoje įvesti maksimaliai skiriami 64 simboliai. & N & NaN & NaN \\ \hline
NFR & NFR 26. Telefono numeriui rezervacijos formoje įvesti maksimaliai skiriama 15 simbolių. & N & NaN & NaN \\ \hline
NFR & NFR 27. Svečių skaičiui rezervacijos formoje įvesti maksimaliai skiriami 3 simboliai. & N & NaN & NaN \\ \hline
NFR & NFR 28. Orų temperatūra rodoma temperatūros matavimo vienetu – celsijumi.  & N & NaN & NaN \\ \hline
NFR & NFR 29. Rezervacijos/užsakymo/sutarties numeris pateikiamas vienetu tikslumu. & N & NaN & NaN \\ \hline
NFR & NFR 30. Įrangos dydžiai - europietiški. Vaizduojami vienetų tikslumu.  & M & Patikslinama savoka ,,europietiški" į ,,metrinė sistema" &  Įrangos dydžiai užrašomi metrine sistema. Vaizduojami vienetų tikslumu. \\ \hline
NFR & NFR 31. Keičiant naršyklės dydį, tinklalapis vaizdą pritaiko automatiškai. & N & NaN & NaN \\ \hline
NFR & NFR 32. Sistema turi veikti bent 97proc. laiko, t.y. maksimalus leidžiamas sistemos nedarbo laikas yra 43 minutės 12 sekundės per parą. & N & NaN & NaN \\ \hline
NFR & NFR 33. Registruojant naują esybę sistema turi patikrinti, ar: 
  NFR 33.1. Esybės įvestas elektroninis paštas yra tinkamo formato ir anksčiau neregistruotas. 
 NFR 33.2. Esybės sugalvotas slaptažodis yra tinkamo formato ir saugus. & M & Neaiški savoka ,,esybė", pakeičiam į ,,vartotojas" ir apjungiam reikalavimą į vieną dėl aiškumo ir konkretumo & Registruojant naują vartotoją sistema turi patikrini ar pateiktas elektroninis parašas yra tinkamo formato ir dar nėra registruotas sistemoje ir slaptažodžio formatas tinkamas. \\ \hline
NFR & NFR 34. Esybei prisijungiant prie sistemos, sistema turi patikrinti, ar: 
 NFR 34.1. Esybės įvestas elektroninis paštas ir slaptažodis yra tinkamo formato. 
 NFR 34.2. Esybės įvestas elektroninis paštas ir slaptažodis yra duomenų bazėje. & M
& Neaiški savoka ,,esybė", pakeičiam į ,,vartotojas" ir apjungiam reikalavimą į vieną dėl aiškumo ir konkretumo & Vartotojui prisijungiant sistema patikrina ar elektroninio parašo ir slaptažodžio formatai tinkami. Sistema taip pat patikrina ar elektroninis parašas ir slaptažodis yra duomenų bazėje. \\ \hline
NFR & NFR 35. Esybei rezervuojant paslaugą sistema turi patikrinti, ar: 
NFR 35.1. Esybės įvestas vardas, pavardė, elektroninis paštas yra tinkamos reikšmės ir formato. 
NFR 35.2. Esybės įvestas telefono numeris yra tinkamo formato. & M & Neaiški savoka ,,esybė", pakleičiam į ,,vartotojas" ir apjungiam reikalavimą į vieną dėl aiškumo ir konkretumo & 
Vartotojui rezervuojant paslaugas sistema turi patikrinti įvesto vardo, pavardės, el. pašto ir telefono numerio formatus. \\ \hline
NFR & NFR 36. Esybei atlikus rezervaciją/užsakymą sistema turi patikrinti, ar rezervacijos/užsakymo numeris yra unikalus.  & N & NaN & NaN \\ \hline
NFR & NFR 37. Modifikuojama tinklalapio atsarginė kopija po kiekvieno informacijos atnaujinimo apie slidinėjimo kurortą, orų prognozes, slidinėjimo trasas, įrangą, apgyvendinimo įstaigą, jų užimtumą bei po kiekvienos esybės registracijos ir įvestos informacijos pakeitimo.  & N & NaN & NaN \\ \hline
NFR & NFR 38. Bandant pildyti laukus ne pagal pateiktus reikalavimus, užklausa negali būti įvykdyta.  & N & NaN & NaN \\ \hline
NFR & NFR 39. Sistemoje turi būti įdiegtos apsaugos priemonės nuo duomenų sugadinimo, praradimo, klaidingų duomenų įvedimo į duomenų bazę. & N & NaN & NaN \\ \hline
NFR & NFR 40. Po kiekvienos sėkmingos operacijos pakeitimai turi būti išsaugomi duomenų bazėje. & N & NaN & NaN \\ \hline
NFR & NFR 41. Nepavykus prisijungti arba negavus duomenų iš duomenų bazės, sistema turi informuoti esybę parodydama klaidos pranešimą. & M & Pakeičiam ,,esybės" į ,,vartotojas" dėl aiškumo &  Nepavykus prisijungti arba negavus duomenų iš duomenų bazės, sistema turi informuoti vartotoją parodydama klaidos pranešimą. \\ \hline
	NFR & 4.2.5. Našumo reikalavimai
NFR 42. Didžiausia leistina tinklalapio sistemos apkrova yra 1000 esybių, prisijungusių vienu metu.
 sekundės. & M & Pakeistas neaiškus daiktavardis & Didžiausia leistina sistemos apkrova yra 1000 vartotojų, prisijungusių vienu metu.\\ \hline
	NFR & 
NFR 43. Tinklalapio didžiausias leistinas reakcijos laikas, neįvertinant interneto greičio, turi būti ne didesnis kaip 2 sekundė. & M & Supaprastintas reikalavimo formulavimas & Sistemos reakcijos laikas turi neviršinti 2 sekundžių\\ \hline
	NFR & 
NFR 44. Užklausos vykdymo laikas turi būti ne didesnis nei 3 sekundės. & M & Nekonkretus reikalavimas & Užklausa turi būti įvygdyta per ne daugiau nei 3 sekundes nuo užklausos gavimo\\ \hline
	NFR & 
NFR 45. Konkrečios slidinėjimo trasos, įrangos, kambario, jų užimtumo paieškai duomenų bazėje turi būti sugaišta ne ilgiau nei 3 & X & Duplikuoja NFR 44 & \\ \hline
	NFR & 4.3. Diegimo reikalavimai
4.3.1. Ruošinio reikalavimai
NFR 46. Tinklalapis pasiekiamas prisijungiant iš bet kurio IP adreso. & M & Patikslintas neaiškus reikalavimas & Sistema neblokuoja jokių IP adresų\\ \hline
	NFR & 
4.3.3. Pradinio DB kaupimo reikalavimai
NFR 47. Pradinėje sistemoje turi būti administratoriaus prisijungimo duomenys. & N & - & -\\ \hline
	NFR & 
NFR 48. Pasirinkimų lentelė turi turėti bent 5 pradines užpildytas eilutes su informacija apie slidinėjimo trasas, įrangą, kambarius. Šią informaciją įveda įgaliotas įmonės administratorius naudodamasis administratoriaus interfeisu. & N & - & -\\ \hline
	NFR & 4.3.4. Sistemos įsisavinamumo reikalavimai
NFR 49. Sistema turi funkcionuoti lietuvių ar anglų kalbomis. & N & - & -\\ \hline
	NFR & 
NFR 50. Įmonės darbuotojai turi būti apmokinami naudotis sistema.& N & - & -\\ \hline
	NFR & 
NFR 51. Stengiamasi išlaikyti tinklalapį suprantamą be vartojimo instrukcijų, t.y. negali būti klaidinančių nuorodų, negali būti vartojami įmantrūs žodžiai, kad nekiltų nesusipratimų. & N & - & -\\ \hline
	NFR & 4.4. Aptarnavimo reikalavimai
NFR 52. Pakeitimai turi būti įvykdyti ne vėliau nei per 7 darbo dienas po sėkmingo testavimo. & N & - & -\\ \hline
	NFR & 
NFR 53. Visi esybės atliekami veiksmai tinklalapyje turi būti sekami ir saugomi laikinoje duomenų bazėje tam, kad jei esybė prisijungimo sesijos metu atrastų tinklalapio spragą, tai visi jo atlikti veiksmai, kurie galėjo tai sukelti, būtų išsiųsti kaip klaidos pranešimas ir darbuotojai, atsakingi už tinklalapio sklandų veikimą, galėtų išanalizuoti spragą ir ją panaikinti. & M & Pakeistas neaiškus daiktavardis ir sutrumpinta sakinio formuluotė & Visi vartotojo atliekami veiksmai sistemoje turi būti sekami ir saugomi laikinoje duomenų bazėje tam, kad jei vartotojas sesijos metu atrastų sistemos spragą, visi jo atlikti veiksmai būtų išsiųsti kaip klaidos pranešimas\\ \hline
	NFR & 
NFR 54. Pastebėtos ar esybių praneštos klaidos turi būti ištaisytos per 5 darbo dienas. & M & Pakeistas neaiškus daiktavardis & Pastebėtos ar vartotojų praneštos klaidos turi būti ištaisytos per 5 darbo dienas\\ \hline
	NFR & 
NFR 55. Į esybės atsiųstus laiškus su pastebėjimais ir skundais reikia atsakyti per 3 darbo dienas. & M & Pakeistas neaiškus daiktavardis & Į vartotojo atsiųstus laiškus su pastebėjimais ir skundais reikia atsakyti per 3 darbo dienas\\ \hline
	NFR & 
NFR 56. Esybės neturi patirti diskomforto susijusio su sistemos atnaujinimu. Jei dėl planuojamo atnaujinimo ar sistemos taisymo reikės trumpam sustabdyti sistemos veiklą, esybės turi būti iš anksto įspėtos ne mažiau nei prieš 1 dieną. & M & Pakeistas neaiškus daiktavardis & Vartotojai neturi patirti nepatogumo susijusio su sistemos atnaujinimu. Jei dėl planuojamo atnaujinimo ar sistemos taisymo reikės trumpam sustabdyti sistemos veiklą, vartotojai turi būti iš anksto įspėti ne mažiau nei prieš 1 dieną\\ \hline


 



\end{longtable}
	
%\end{table}

\hbox to 150mm{\hbox to 149.99mm{}}

\section{Užduotys}

\section{Reikalavimų specifikacijos, dalykinės srities modelio ir užduočių diagramos peržiūros rezultatai}

\section{Išvada}

\section{Asmeninis darbo indėlis}

\section{Žodynas}

\end{document}
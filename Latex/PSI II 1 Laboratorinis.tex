\documentclass[oneside]{VUMIFPSkursinis}
\usepackage{algorithmicx}
\usepackage{algorithm}
\usepackage{algpseudocode}
\usepackage{amsfonts}
\usepackage{float}
\usepackage{amsmath}
\usepackage{bm}
\usepackage{caption}
\usepackage{color}
\usepackage{float}
\usepackage{graphicx}
\usepackage{listings}
\usepackage{subfig}
\usepackage{tabularx}
\usepackage{wrapfig}
\newcolumntype{P}[1]{>{\centering\arraybackslash}p{#1}}
\usepackage[%  
    colorlinks=true,
    linkcolor=black
]{hyperref}
\university{Vilniaus universitetas}
\faculty{Matematikos ir informatikos fakultetas}
\department{Programų sistemų katedra}
\papertype{Programų sistemų inžinerija II laboratorinis darbas I}
\title{Reikalavimų apibrėžimas}
\titleineng{Requirements specification}
\status{2 kurso 3 grupės studentai}
\secondauthor{Justas Tvarijonas}  
\thirdauthor{Greta Pyrantaitė}
\fourthauthor{Rytautas Kvašinskas}
\author{Tomas Kiziela}

\supervisor{Audronė Lupeikienė, M. Darbuot., Dr.}
\date{Vilnius – \the\year}


\bibliography{bibliografija}

\begin{document}
\maketitle
\tableofcontents

\section{Įžanga}
Mūsų komanda gavo kitos komandos pirmame semestre(PSI I) paruoštą slidinėjimo kurorto projektą. Šiame darbe sieksime toliau plėtoti ir keisti šį projektą. Toliau vystatnt projektą keisis daugumas dalių. Siekiant padaryti gerą produktą kai kurios  dalys bus pašalindos ir pridėtos naujos. Pirmame semestre projektuojant dėmesys buvo skiriamas klasikinei projektavimo paradigmai. Šiame darbe projektą rašysime pasinaudodami ICONIX principu, projektuojant dėmesys bus skiriamas GUI sudarymo ir iš to išplauks sistemos projektavimas ir sandara. 

\section{Reikalavimai}
Šioje dalyje bus pateikti funkciniai bei nefunkciniai reikalavimai sistemai. Stengsimės prisilaikyti ,,užsakovų"(grupės iš kurios gavome jų darbą) reikalavimus tačiau siekiant sukurti geresnę sistemą pridėsime kaikuriuos savo sugalvotus reikalavimus arba ignoruosime mums pateiktus reikalavimus. 

\subsection{Reikalavimų pataisymai}
	\begin{itemize}
		\item{X - ištrintas}
		\item{M - pakeistas}
	\end{itemize}


\begin{table}[htbp]


\begin{tabularx}{1\textwidth}{  |P{1cm}|X|c|p{3.5cm}|X| }  \hline
	Nr. & Pradiniai reikalavimas&  Pakeitimo tipas & Klaidos aprašas  & Naujas reikalavimas \\ \hline
	FR & ,,Žetonas" seka vartotojo laiką praleistą trasoje & M & Pakeistas neaiškus daiktavardis & Sekimo prietaisas seka vartotojo laiką praleista trasoje \\ \hline	
  FR & Vartotojo prieeigos prie pramogų prieinamumas nustatomas naudojant pirštų antspaudą & M & Sukonkretintas abstraktus reikalavimas & Vartotojui norint pasinaudoti kavinės paslaugomis naudojamas piršto antspaudas \\ \hline
  FR &  Sistema leidžia vartotojui už paslaugas atsiskaityti e-bankininkyste & N & - & - \\ \hline
	FR & ,,Žetonas" skaičiuoja greičiausią laiką, per kurį vartotojas įveikia trasą & M & Pakeistas neaiškus daiktavardis & Sekimo prietaisas fiksuoją greičiausią laiką, per kurį vartotojas įveikė trasą \\ \hline
	FR & Vartotojo trasų laikai rodomi internetinėje aplikacijoje & N & NaN & NaN \\ \hline
	FR & Sistema seka vartotojų poziciją specialaus žetono pagalba, kurį gauna kiekvienas vartotojas apsilankęs kurorte(Vieta sekama tik gavos vartotojo sutikimą) & M & Reikalavimas skliausteliuose turėtų būti perkeltas į nefunkcinius (pvz. saugumo) raiklavimus & Sistema seka vartotojo poziciją specialaus sekimo prietaiso pagalba, kurį gauna kiekevienas apsilankęs kurorte \\ \hline
	FR & 3.3.1. Parašyti ataskaitą
FR 45. Pradiniai duomenyswhy 
FR 45.1. Ataskaitos pavadinimas;
FR 45.2. Ataskaitos turinys;
FR 46. Veiksmai
FR 46.1. Darbuotojas meniu juostoje paspaudžia ant mygtuko “Reports”;
FR 46.2. Darbuotojas parašo ataskaitos pavadinimą bei turinį;
FR 46.3. Darbuotojas paspaudžia mygtuką “Send”
FR 48. Reikalavimai
FR 48.1. Darbuotojas turi būti prisijungęs;
FR 48.2. Pavadinimas bei tema neturi būti tušti;
FR 48.3. Pavadinimo ilgis neturi viršyti 50 simbolių;
FR 48.4. Turinio ilgis neturi viršyti 500 simbolių;
FR 49. Rezultatai
FR 49.1. Darbuotojui pranešama, jog ataskaita sėkmingai išsiųstas. & M & trying to get gold from shit & Vartotojas turi galėti parašyti ataskaitą\\ \hline


	
	


\end{tabularx}

	
\end{table}


\begin{table}[htbp]


\begin{tabularx}{1\textwidth}{  |P{1cm}|X|c|p{3.5cm}|X| }  \hline

Nr. & Pradiniai reikalavimas&  Pakeitimo tipas & Klaidos aprašas  & Naujas reikalavimas \\ \hline
FR & 
FR 93. Pradiniai duomenys: 
FR 93.1. Objekto kategorija(-os). 
FR 94. Veiksmai: 
FR 94.1. Meniu juostoje paspaudžiama ant mygtuko „Set the object state”;
 FR 94.2. Iš objektų kategorijų sąrašo (kambarys, slidinėjimo trasa, slidinėjimo įranga) pasirenkama viena kategorija;
 FR 94.3. Parodomas sąrašas objektų, priklausančių tai kategorijai;
 FR 94.4. Pasirenkamas vienas objektas; FR 94.5. Pasirenkama data;
 FR 94.6. Pažymima būsena, paspaudžiant ant mygtuko „Set the state”.
 FR 95. Alternatyvūs scenarijai:
 FR 95.1. Nutrūko ryšys su duomenų baze.
 FR 96. Reikalavimai: 
FR 96.1. Administratorius turi būti prisijungęs.
 FR 97. Rezultatai:
 FR 97.1. Duomenų bazėje pažymima objekto būseną. 
& M & Reikalavimai parašyti nekorektiškai, reikalinga sukonkretinti ir suabstraktinti reikalavimus įvieną reikalavimą &
Sistema administratoriui suteikia galimybę stebėti duombazės objektų būseną. Sistemoje administratorius gali stebėti arba keisti objektų būsenas. Pasirinkus veikmsą parodomas objektų kategorijų sąrašas. Nutrūkus ryšiui su duombaze yra išsaugoma paskutinė darbo būsena.\\ \hline




\end{tabularx}

	
\end{table}

\subsection{Slidininko sekimas realiu laiku}

\begin{table}[htbp]

\begin{tabularx}{1\textwidth}{ |P{2.5cm}|X|P{3cm }| }  \hline
	Nr. & Reikalavimai & Prioritetas \\ \hline
	FR 1 & Sistema leidžia vartotojui už paslaugas atsiskaityti e-bankininkyste & 10 \\ \hline
	FR 2 & Vartotojo prieeigos prie pramogų prieinamumas nustatomas naudojant pirštų antspaudą & 8 \\ \hline
	
	
	
	FR 3& Sistema seka vartotojų poziciją specialaus žetono pagalba, kurį gauna kiekvienas vartotojas apsilankęs kurorte(Vieta sekama tik gavos vartotojo sutikimą) & 8  \\ \hline
	FR 4 & ,,Žetonas" seką vartotojo laiką praleista trasoje & 8 \\ \hline
	FR 5 & ,,Žetonas" skaičiuoją greičiausią laiką per kurį vartotojas įveikia trasą &  8 \\ \hline
	FR 6 & Vartotojo trasų laikai rodomi internetinėje aplikacijoje & 7 \\ \hline 
	

\end{tabularx}

	
\end{table}





 
	

\section{Užduotys}

\section{Reikalavimų specifikacijos, dalykinės srities modelio ir užduočių diagramos peržiūros rezultatai}

\section{Išvada}

\section{Asmeninis darbo indėlis}

\section{Žodynas}


	
	





\end{document}
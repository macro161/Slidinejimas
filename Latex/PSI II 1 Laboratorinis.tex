\documentclass[oneside]{VUMIFPSkursinis}
\usepackage{algorithmicx}
\usepackage{algorithm}
\usepackage{algpseudocode}
\usepackage{amsfonts}
\usepackage{float}
\usepackage{amsmath}
\usepackage{bm}
\usepackage{caption}
\usepackage{color}
\usepackage{float}
\usepackage{graphicx}
\usepackage{listings}
\usepackage{subfig}
\usepackage{tabularx}
\usepackage{wrapfig}
\newcolumntype{P}[1]{>{\centering\arraybackslash}p{#1}}
\usepackage[%  
    colorlinks=true,
    linkcolor=black
]{hyperref}
\university{Vilniaus universitetas}
\faculty{Matematikos ir informatikos fakultetas}
\department{Programų sistemų katedra}
\papertype{Programų sistemų inžinerija II laboratorinis darbas I}
\title{Reikalavimų apibrėžimas}
\titleineng{Requirements specification}
\status{2 kurso 3 grupės studentai}
\author{Matas Savickis}
\secondauthor{Justas Tvarijonas}  
\thirdauthor{Greta Pyrantaitė}
\fourthauthor{Rytautas Kvašinskas}   
\supervisor{Audronė Lupeikienė, M. Darbuot., Dr.}
\date{Vilnius – \the\year}


\bibliography{bibliografija}

\begin{document}
\maketitle
\tableofcontents

\section{Įžanga}
Mūsų komanda gavo kitos komandos pirmame semestre(PSI I) paruoštą slidinėjimo kurorto projektą. Šiame darbe sieksime toliau plėtoti ir keisti šį projektą. Toliau vystatnt projektą keisis daugumas dalių. Siekiant padaryti gerą produktą kai kurios  dalys bus pašalindos ir pridėtos naujos. Pirmame semestre projektuojant dėmesys buvo skiriamas klasikinei projektavimo paradigmai. Šiame darbe projektą rašysime pasinaudodami ICONIX principu, projektuojant dėmesys bus skiriamas GUI sudarymo ir iš to išplauks sistemos projektavimas ir sandara. 
\section{Reikalavimai}

\section{Struktūrinis dalykinės srities modelis}

\section{Užduotys}

\section{Reikalavimų specifikacijos, dalykinės srities modelio ir užduočių diagramos peržiūros rezultatai}

\section{Išvada}

\section{Asmeninis darbo indėlis}

\section{Žodynas}


	
	





\end{document}